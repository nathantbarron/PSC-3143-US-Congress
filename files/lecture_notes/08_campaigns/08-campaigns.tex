\documentclass[10pt]{article} 
\usepackage{geometry, hyperref, booktabs}
\geometry{tmargin=1in, bmargin=1in, lmargin=2.5in, rmargin = 1in}  

\title{\textbf{U.S. Congress}\\\textit{Congressional Campaigns}}
\author{Nathan Barron}
\date{Fall 2024}

\begin{document}
	\maketitle
	\tableofcontents
	\vspace{.25in}
	\hrule
	\vspace{.25in}
	
	
	\section{Do (Congressional) campaigns matter?}
	
	
	\subsection{Spending}
	Three historical views: 
	\begin{itemize}
		\item Challenger spending matters and incumbent spending does not\footnote{ Jacobson 1985, 1990, 2013; Abramowitz 1988; Ansolabehere \& Gerber 1994}
		\item Marginal returns are greater for challengers than for incumbents, but not as stark as previously assumed\footnote{Bartels 1991,
			Goidel \& Gross 1994, Kenney \& McBurnett 1994}
		\item Spending by both incumbents and challengers has about the same effect or that incumbent spending is actually more potent\footnote{Green \& Krasno 1988, 1990; Erikson \& Palfrey 1998; Goldstein \& Freedman 2000}
	\end{itemize}
	
	Spending can be particularly effect in changing who votes, particularly among low-information voters, partisans, and those who are economically dissatisfied.\footnote{Schuster, Steven Sprick. 2020. ``Does Campaign Spending Affect Election Outcomes? New Evidence from Transaction-Level Disbursement Data." \textit{The Journal of Politics}. 82(4): 1502-1515.}
	
	\subsection{Persuasion}
	
	Broockman and Kalla\footnote{Kalla, Joshua L., and David E. Broockman. ``The Minimal Persuasive Effects of Campaign Contact in General Elections: Evidence from 49 Field Experiments.” \textit{American Political Science Review} 112, no. 1 (2018): 148–66. https://doi.org/10.1017/S0003055417000363.} find that campaigns are not effective in persuading potential voters. Recent literature suggests that there might be a small ability to persuade voters in cases where campaigns calibrate advertisements with experiments.\footnote{Hewitt, Luke, David Broockman, Alexander Coppock, Ben M. Tappin, James Slezak, Valeria Coffman, Nathaniel Lubin, and Mohammad Hamidian. “How Experiments Help Campaigns Persuade Voters: Evidence from a Large Archive of Campaigns’ Own Experiments.” \textit{American Political Science Review}, 2024, 1–19. https://doi.org/10.1017/S0003055423001387.}
	
	\subsection{Information and knowledge}
	
	\begin{itemize}
		\item Campaigns are especially helpful for challengers and early-career incumbents; improving name-recognition
		\item Particularly relevant to spending (see table below)
	\end{itemize}
	
	
	\begin{table}[h]
		\centering
		\begin{tabular}{lccc}
			\toprule
			& Total spending  & 2 months prior & Election day \\ 
			& during election  & to Election day & Election day \\ \midrule
			\\[-.8em]
			\textbf{DV = Recall}\\
			Incumbents & $>$\$400k & 35 & 50\\
			Incumbents & $<$\$400k & 25 & 40 \\
			Challengers & $>$\$400k & 20 & 40 \\
			Challengers & $<$\$400k & 5 & 15 \\ \\
			\textbf{DV = Recognize} \\
			Incumbents & $>$\$400k & 80 & 90 \\
			Incumbents & $<$\$400k & 80 & 90 \\
			Challengers & $>$\$400k & 50 & 80\\ 
			Challengers & $<$\$400k & 25 & 45\\ \bottomrule
		\end{tabular}
		\caption{Percentage (\%) of Survey Respondents who could Recall or Recognize Congressional Candidates by Spending, Incumbency Status. (Source: Jacobson 2006)}
	\end{table}
	
	\subsection{Agendas, framing \& priming}
	\begin{itemize}
		\item Agendas are the topics that candidates and parties discuss.
		\item Framing is \textit{how} to think about certain policies, events, etc. Importantly, framing isn't \textit{what} to think about those policies. For example, the Republican and Democratic parties differ in how to frame immigration issues: national security or humanitarianism. Framing is about how to process information.
		\item Priming is bringing certain topics to the forefront by increasing those topics' perceived relevance.
	\end{itemize}
	
	
	\subsection{Voter turnout}
	
	
	\begin{table}[h]
		\centering
		\begin{tabular}{lc}
			\toprule
			GOTV & Increased \\ 
			Strategy& turnout   \\ 
			\midrule
			Canvassing & + 2.5\\
			Volunteer phone bank & + 1.9 \\
			Commercial phone bank & + 1.0\\
			Robocalling & + 0.0 \\
			Direct mail & + 0.2\\
			Text messaging & + 4.1\\
			\bottomrule
		\end{tabular}
		\caption{Percentage Point Increase in Voter Turnout by GOTV Strategy (Source: Jacobson 2015)}
	\end{table}
	
	
	
\end{document}
