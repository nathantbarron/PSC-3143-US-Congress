\documentclass[10pt]{article} 
\usepackage{geometry, hyperref, booktabs}
\geometry{tmargin=1in, bmargin=1in, lmargin=2.5in, rmargin = 1in}  

\title{\textbf{U.S. Congress}\\\textit{Fundraising}}
\author{Nathan Barron}
\date{Fall 2024}

\begin{document}
	\maketitle
	\tableofcontents
	\vspace{.25in}
	\hrule
	\vspace{.25in}
	
	
	\section{Major Supreme Court Rulings}
	
	\subsection{\textit{Buckley v. Valeo} (1976)}
	\begin{itemize}
		\item Congress passes the Federal Election Campaign Act of 1971 as an anti-corruption measure after Watergate
		\item FECA aimed to regulate who could donate how much to campaigns and how campaigns could spend that money
		\item SCOTUS determined that political donations carry the weight of political speech and are thus given First Amendment rights
		\item SCOTUS determined that the FEC can regulate individual contributions to candidate campaigns; while these donations are free speech, anti-corruption is a compelling government interest that must be counterbalanced
		\item SCOTUS determined that the FEC cannot regulate a campaign's total expenditures or money self-raised; the government's interest in anti-corruption is not sufficiently advanced by these regulations
	\end{itemize}
	
	\subsection{\textit{McConnell v. FEC} (2003)}
	\begin{itemize}
		\item Congress passes the Bipartisan Campaign Reform Act of 2002 (or the McCain-Feingold Act) to regulate ``soft money" transfers from source-to-source that bypass giving limits and place additional restrictions on ``electioneering" activities
		\item SCOTUS ruled that Congress possessed the authority to regulate soft money and electioneering with 60 days of the election
	\end{itemize}
	
	\subsection{\textit{Citizens United v. FEC} (2010)}
	\begin{itemize}
		\item The conservative group Citizens United published a film \textit{Hillary: The Movie} during the 2008 election cycle where Hillary Clinton ran against Barrack Obama for the Democratic nomination for President
		\item The FEC attempted to enforce electioneering and finance restrictions on Citizens United for engaging in election-related expenditures
		\item SCOTUS determined that the First Amendment fully protects \textit{independent} expenditures as political speech unrelated to the purpose of anti-corruption
		\item This case led to the proliferation of independent PACs -- known as Super PACs -- and dark money
	\end{itemize}
	
	
	\section{Other information}
	
	\subsection{Where does the money come from?}
	
	\begin{table}[h]
		\centering
		\begin{tabular}{lcccc}
			\toprule
			& Small Individual & Large Individual & &   \\
			& Giving           & Giving & PACs & Self-financed\\ \midrule
			House Dems & 46.4 & 18.4 & 21.2 & 6.9\\
			House Reps & 36.2 & 14.9 & 32.6 & 0.2\\
			Senate Dems & 52.4 & 25.7 & 9.7 & 0.0\\
			Senate Reps & 41.2 & 19.1 & 14.7& 3.6\\ \bottomrule
		\end{tabular}
		\caption{Percentage (\%) of Campaign Funds by Origin, 2024 Congressional Races (Source: OpenSecrets.org)}
	\end{table}
	
	\subsection{Slides}
	
	\textit{Be sure to review the slides and be familiar with interpretations for the figures therein.}
	
\end{document}
