\documentclass[10pt]{article} 
\usepackage{geometry}
\usepackage{hyperref}
\usepackage{booktabs, rotating}
\geometry{tmargin=1in, bmargin=1in, lmargin=2.5in, rmargin = 1in}  

\title{\textbf{U.S. Congress}\\\textit{Historical and Constitutional Origins}}
\author{Nathan Barron}
\date{Fall 2024}

\begin{document}
\maketitle
\tableofcontents
\vspace{.25in}
\hrule
\vspace{.25in}

\section{The Article I Institution}

\subsection{Section-by-Section Outline}

\begin{enumerate}
    \item Vesting legislative power
    \item Apportionment; qualifications; impeachment (HOUSE)
    \item Apportionment; qualifications; trial of impeachment (SENATE)
    \item Elections by states
    \item Rules (seating, chamber rules, records, adjournment)
    \item Compensation; arrest; holding single office
    \item Origination, passage of bills
    \item Expressed powers
    \item Restrictions on Congress
    \item Restrictions on States
\end{enumerate}

\subsection{Important Clauses}

\textbf{Commerce clause}\newline \textit{[The Congress shall have Power . . . ] To regulate Commerce with foreign Nations, and among the several States, and with the Indian Tribes; . . .}\footnote{Visit the following links for more on the \href{https://constitution.congress.gov/browse/essay/artI-S8-C3-1/ALDE_00013403/}{Commerce Clause}, \href{https://constitution.congress.gov/browse/essay/artI-S8-C3-2/ALDE_00013404/}{the Meaning of ``Commerce"}, \href{https://constitution.congress.gov/browse/essay/artI-S8-C3-3/ALDE_00013405/}{the Meaning of ``Among the Several States"}, \href{https://constitution.congress.gov/browse/essay/artI-S8-C3-4/ALDE_00013406/}{the Meaning of ``Regulate"}, \href{https://constitution.congress.gov/browse/article-1/section-8/clause-3/}{various other tidbits}.} \\




\noindent \textbf{Elastic clause} (or, the necessary and proper clause) \newline \textit{[The Congress shall have Power . . . ] To make all Laws which shall be necessary and proper for carrying into Execution the foregoing Powers, and all other Powers vested by this Constitution in the Government of the United States, or in any Department or Officer thereof.}\footnote{Visit the following links for more on the \href{https://constitution.congress.gov/browse/essay/artI-S8-C18-1/ALDE_00001242/}{Elastic Clause}, \href{https://constitution.congress.gov/browse/essay/artI-S8-C18-2/ALDE_00001237/}{Historical Background}, \href{https://constitution.congress.gov/browse/essay/artI-S8-C18-3/ALDE_00001238/}{Early Doctrine},  \href{https://constitution.congress.gov/browse/essay/artI-S8-C18-4/ALDE_00001239/}{Doctrinal Development}, \href{https://constitution.congress.gov/browse/essay/artI-S8-C18-5/ALDE_00001240/}{Modern Doctrine}, and the \href{https://constitution.congress.gov/browse/essay/artI-S8-C18-6/ALDE_00001241/}{Meaning of ``Proper"}.}





\subsection{Relevant Amendments}

\begin{itemize}
    \item Amendments 1-10: The Bill of Rights
    \item 14th Amendment: Apportionment; Legislation to enforce due process, equal protection
    \item 17th Amendment: Direct election of senators
    \item 27th Amendment: Restriction on compensation
\end{itemize}

\subsection{Types of Powers}

\textbf{Expressed powers} are those powers that are specifically identified in the Constitution (see Article I, Section 8). \textbf{Implied powers} are those ``necessary to effectuate powers enumerated in the Constitution". You can read more about expressed and implied powers \href{https://constitution.congress.gov/browse/essay/artI-S1-3-3/ALDE_00013292/}{here} (as well as resulting and inherent, though we won't discuss those). 






\begin{sidewaystable}[h]
\section{Contrast: The Articles of Confederation}
    \centering
\begin{tabular}{lll}
    Issue           &       Articles of Confederation       &       US Constitution             \\
    \midrule
    \textbf{General structure} &     Unicameral; 1 vote per state    &       Bicameral; 1 vote per member \\
    \\
    \textbf{Apportionment} & Each state had 2-7 members;  & House membership based on population per state;  \\
     & one 1 vote per state & Senate membership was equally apportioned to states \\ \\
     \textbf{Terms} & One year term; 3 term limit & House term is 2 years; Senate term is 6 years; \\
     & & No term limits\\ \\
     \textbf{Mode of election} & Elected by, paid by, and & House is popularly elected; Senate is appointed \\
     & recallable by the state legislature & by state legislature; paid by US Treasury; \\
     & & no recall; both chambers judge their own elections \\ \\
     \textbf{Internal structure} & Not specified & Both given latitude to write/enforce \\
     && rules; arrest and compel members\\ \\
     \textbf{Legislative powers} & Approve foreign treaties by states; & Lay and collect taxes \\
     & declare war; recommend taxes to states; & ratify treaties; confirm exec. nominess;\\
     & adjudicate interstate disputes; & regulate interstate and foreign commerce;\\
     & regulate mail between states; & declare war; provide post offices and roads;\\
     & regulate coinage, weights, and measures; & regulate coinage, weights, and measures;\\
     & raise/equip army and navy & raise/equip army and navy\\
     && provide for inferior federal courts;\\
     &&control federal district \\ \bottomrule
\end{tabular}
\end{sidewaystable}



\end{document}
