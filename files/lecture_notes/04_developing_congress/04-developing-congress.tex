\documentclass[10pt]{article} 
\usepackage{geometry}
\geometry{tmargin=1in, bmargin=1in, lmargin=2.5in, rmargin = 1in}  

\title{\textbf{U.S. Congress}\\\textit{A Developing Congress}}
\author{Nathan Barron}
\date{Fall 2024}

\begin{document}
\maketitle
\tableofcontents
\vspace{.25in}
\hrule
\vspace{.25in}

\section{Five Periods of Congressional Systems}

The following typology is taken from Stewart's \textit{Analyzing Congress}.

\subsection{Experimental (1789-1812)}
\begin{itemize}
    \item Jeffersonian model of legislative workflow 
    \item Most of the work occurred in the whole chamber
    \item Committees were mostly ad hoc and requested to report back to chamber
    \item Speaker was mostly a moderator; he could also appoint members to ad hoc committees
    \item ``Legislative" influencers were often outside of Congress
\end{itemize}



\subsection{Democratizing (1820-1860)}
\begin{itemize}
    \item Leadership of Henry Clay
    \item The solidification of party powers
    \item Establishment of standing committees
    \item Increased workload demands (post-war pensions)
    \item Increased interbranch competition
    \item Oversight of executive branch
\end{itemize}



\subsection{Reconstruction (1865-1896)}
\begin{itemize}
    \item Firmer, more regional partisan affiliations
    \item Greater party homogeneity in the electorate
    \item Partisan structuring of committees
    \item Stronger procedural protections for majority in the House (Reed's Rules), including removal of disappearing quorum, removal of dilatory motions, creation of rules committee and scheduling and special orders
\end{itemize}



\subsection{Textbook Era (1912-1968)}
\begin{itemize}
    \item Partisan realignment of 1890s
    \item Australian (secret) ballot and primaries electoral reforms
    \item Speaker Canon's iron-fist rule over the Rules Committee and schism in the Republican party
    \item Canon revolt: reduced power of the Speaker
    \item Creation of the Committee on Committees
    \item Democratic emphasis on party consensus (2/3 caucus rule)
    \item Republican rejection of the caucus style
\end{itemize}



\subsection{Candidate-Centered Congress (1973-?)}
\begin{itemize}
    \item 1973 Reform: Stronger more developed committee and subcommittee system
    \item Reforms designed to benefit the majority party's agenda
    \item Increased mass polarization
    \item More deference to party leaders for agenda-setting and party discipline
    \item Technology and the opportunity for personal brand-making 
    \item The cost of elections and the value of fundraising
\end{itemize}



\end{document}
