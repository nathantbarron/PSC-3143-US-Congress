\documentclass[10pt]{article} 
\usepackage{geometry, hyperref}
\geometry{tmargin=1in, bmargin=1in, lmargin=2.5in, rmargin = 1in}  

\title{\textbf{U.S. Congress}\\\textit{Apportionment and Redistricting}}
\author{Nathan Barron}
\date{Fall 2024}

\begin{document}
	\maketitle
	\tableofcontents
	\vspace{.25in}
	\hrule
	\vspace{.25in}
	
	
	\section{Formal Rules of the Game}
	
	Article I, Section 2 of the Constitution provides for how the House of Representatives will go about apportionment: 
	
	\begin{quote}
		\footnotesize
		Representatives ... shall be apportioned among the several States ... according to their respective Numbers, which shall be determined by adding to the whole Number of free Persons, including those bound to Service for a Term of Years, and excluding Indians not taxed, three fifths of all other Persons. The actual Enumeration shall be made within three Years after the first Meeting of the Congress of the United States, and within every subsequent Term of ten Years, in such Manner as they shall by Law direct.
	\end{quote}
	
	Included in this section are provisions for (1) the partial counting of slaves and non-counting of ``Indians not taxed"\footnote{The 3/5 Compromise was repealed by the 14th Amendment. However, the non-counting of ``Indians not taxed" remained.}, (2) apportionment to occur at least every 10 years, and (3) state legislatures to direct the process\footnote{There is a so-called ``Independent State Legislature Theory" that has arisen lately. This misguided Constitutional interpretation argues that state legislatures are the sole power in the redistricting process and that state courts (or others) have no Constitutional authority to reject, revise, or otherwise prevent state legislatures from choosing any redistricting plan. The U.S. Supreme Court ruled in \textit{Moore v. Harper} (2023) that the North Carolina legislature's authority in developing redistricting maps was neither ``exclusive" nor ``independent" and that the legislature was still subject to judicial review under the North Carolina Constitution.}. 
	
	The following are some important recent Supreme Court decisions concerning Congressional redistricting: 
	
	\begin{itemize}
		\item \textit{Rucho v. Common Cause} (2019): Partisan gerrymandering claims are not justiciable because they present a political question beyond the reach of the federal courts.
		\item \textit{Arizona State Legislature v. Arizona Independent Redistricting Commission} (2015): Arizona's use of an independent committee to draw congressional districts is constitutional. 
		\item \textit{Shelby County v. Holder} (2013): The federal government can no longer enforce the Voting Rights Act's preclearance\footnote{Preclearance is the requirement that the U.S. Department of Justice must review and approve changes in electoral requirements (including redistricting maps) prior to state implementation.} requirements on certain areas according to a area definition\footnote{The areas were originally defined as those that had voting tests by November 1, 1964, and had less than 50\% voter turnout in the 1964 Presidential election.} last updated in 1975.
	\end{itemize}
	
	\noindent Of course, there are plenty other legal challenges across the entire country. Check some of them out here: \href{https://redistricting.lls.edu/cases/?sortby=-updated&page=1}{Pending Redistricting Court Cases}. Also see a recent Congressional Research Service (CRS) Legal Sidebar on the \href{https://crsreports.congress.gov/product/pdf/LSB/LSB10639}{Legal Framework for Congressional Redistricting}.
	
	\subsection{The Essential Requirements}
	
	\begin{itemize}
		\item \textbf{Population equality}: Congressional districts are required to have roughly the same number of total population (as the other districts in the state).  In \textit{Tennant v. Jefferson County Commission} (2012), the Court ruled that minor population disparities were allowable in order to achieve selective ``legitimate state interests." Partisanship is not a legitimate state interest (\textit{Harris v. Arizona Independent Redistricting Commission} (2016)). Federal requirement.
		\item \textbf{Minority protection}: The Voting Rights Act\footnote{Reminder: Congress's authority to restrict state's redistricting abilities arises from the 14th Amendment.} "prohibits states or political subdivisions from imposing any voting qualification, practice, or procedure that results in denial or abridgement of the right to vote based on race, color, or membership in a language minority."\footnote{CRS Report IN11618. \url{https://crsreports.congress.gov/product/pdf/IN/IN11618}} The purpose is to prevent the intentional reduction or dilution of minority voting strength. However, states also cannot be drawn with primarily race in mind (\textit{Shaw v. Reno} (1990)). Federal requirement.
		\item \textbf{Compactness}: Congressional districts should aim to minimize the distance between citizens within the district. The ``perfectly compact" district is a circle. At least 31 states require compactness.
		\item \textbf{Contiguity}: Congressional districts should be drawn so that it is possible to travel between any two points in a district without crossing into another (exceptions for water). At least 34 states require contiguity. 
		\item \textbf{Political subdivisions}: Congressional districts should be drawn so as to preserve existing political boundaries (e.g. cities, counties, etc.). At least 31 states require consideration of maintaining political subdivisions. 
	\end{itemize}
	
	\subsection{Additional Considerations}
	
	\begin{itemize}
		\item Partisan composition
		\item Protecting incumbents
		\item Competitiveness
	\end{itemize}
	
	\noindent Read more about other redistricting criteria and the states that require them here: \href{https://www.ncsl.org/elections-and-campaigns/2020-redistricting-criteria}{Redistricting Criteria}.
	
	\section{Who draws the lines?}
	
	The Constitution gives states the ability to create their own district maps. Because state must do this through legislation, state legislatures are assumed to have the power to develop these maps. However, some state legislatures have redirected the task of developing these maps to redistricting commissions, either due to workload management (partisan redistricting commissions) or anti-gerrymandering reforms (independent redistricting commissions). 
	
\subsection{Who was supposed to draw the maps}
\begin{itemize}
    \item 33 Legislature (3 backup commissions, 3 advisory commissions)
    \item 8 Independent Commissions
    \item 3 Partisan Commissions
    \item 6 NA
\end{itemize}
	
\subsection{Who ended up drawing the maps}
	\begin{itemize}
		\item 28 Legislatively-Drawn Maps
		\item 8 Independent Commissions
		\item 1 Partisan Commissions
		\item 6 State Courts\footnote{State courts will often develop maps in the event of stalemate between the legislature and governor, when permitted by the state constitution. Generally these maps are intended to be temporary until a legislative solution can be found; in practice, these maps are often retained.}
		\item 1 Federal Court\footnote{A three-judge panel of the U.S. District Court in north Alabama appointed a Special Master to draw Alabama's Congressional Districts after repeated failure to comply with the VRA.}
		\item 6 NA
	\end{itemize}
	
	
	\section{Application: Measuring Compactness}
	
	Kaufman, Aaron R., Gary King, and Mayya Komisarchik. 2021. ``How to Measure Legislative District Compactness if You Only Know It When You See It." \textit{American Journal of Political Science.} 65(3): 533-550.\footnote{\url{https://gking.harvard.edu/sites/scholar.harvard.edu/files/gking/files/ajps.12603.pdf}}
	
\end{document}
