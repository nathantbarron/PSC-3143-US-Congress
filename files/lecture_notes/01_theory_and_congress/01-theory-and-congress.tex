\documentclass[10pt]{article} 
\usepackage{geometry}
\usepackage{hyperref}
\geometry{tmargin=1in, bmargin=1in, lmargin=2.5in, rmargin = 1in}  

\title{\textbf{U.S. Congress}\\\textit{Theory and Congressional Politics}}
\author{Nathan Barron}
\date{Fall 2024}

\begin{document}
\maketitle
\tableofcontents
\vspace{.25in}
\hrule
\vspace{.25in}

\section{Why science?}
\subsection{Alternative approaches}

There are several ways that we could learn about Congress. The first (and often mistaken for political science) is the \textit{historical} approach. The historical approach's goal is to understand the complex web of events, people, and dilemmas in order to explain the past and contextualize the present and future. The historical approach is event-focused. 

Another approach is the \textit{textbook} approach. The ``textbook Congress" is an introductory but ultimately inadequate framework for students of Congressional politics. The textbook approach focuses on (among other things) original Constitutional design and naive understandings of legislative process (e.g. think Schoolhouse Rock ``I'm just a bill"). 

Thirdly, there is a \textit{normative} approach to learning about Congress that is rooted in ideological conceptions of what a national legislature ought to be. This approach tends to focus on the formation of idealistic visions of collective action while engaging others' sometimes differing idealistic visions (e.g. think about debate-oriented learning). 

While other approaches to learning about Congress have their place, we will approach this class with a scientific approach.



\subsection{The scientific approach}
The \textit{scientific} approach aims to understand phenomena. If the historical approach asks the question ``what happened?", the scientific approach asks ``what happen\textbf{s}?" Importantly, the goal underlying the scientific approach is to generalize beyond specific cases in order to predict future conditions. In this sense, the scientific approach is phenomena-focused (or, concept-focused). 

For example, if I made a \href{https://www.paperairplaneshq.com/planes/catfish.html}{catfish-style paper airplane}\footnote{Check out some cool paper airplane designs here: https://www.paperairplaneshq.com/.} and threw it across the room, I would want to know how far it had flown. But I'll also want to know what will happen if I do it again. If I keep making catfish style airplanes, I need to know how far I would expect it to fly. Does it fly further than the far easier-to-make seagull design? Each throw will be different, the plane might take some damage or the air conditioner might kick on -- but I want to use the past throws to develop expectations about future throws. 

Science relies on this type of generalization of past events to predict future events. Political Scientist Jon Bond characterised the basic premise of studying politics scientifically like this: ``There are basic laws that explain political behavior and these laws can be discovered through the scientific method."\footnote{Bond, Jon. 2007. ``The Scientification of the Study of Politics." \textit{The Journal of Politics.} 69(4): 897-907.} We attempt to discover these laws by observing behavior and testing hypotheses. However, one of the most important roles of the political scientist is to clearly state these laws as \textit{theory}. 



\section{The role of theory in political science}

\subsection{What is theory?}

A \textit{theory} is a systematic explanation of some observable facts. There are three levels of theories: 

\begin{itemize}
    \item \textbf{High theory} (i.e. framework, paradigm): This is a hyper-abstract and generalized explanation of events that transcends disciplines, subject matter, etc. An example of a high theory would be rational choice theory.
    \item \textbf{Mid theory} (i.e. theory): Mid theories are abstracted systematic explanations for a more particular set of actions. Usually a more applied version of high theory, these mid theories are more likely to be what we discuss in this class. An example of a mid theory would be the spatial theory of voting. 
    \item \textbf{Low theory} (i.e. logical derivation): Low theories are usually comprised of a chain of propositions that are rooted in mid theories but applied to specific situations. Low theory is usually the level of theory meant when writing a ``theory" section for an empirical paper. An example of a low theory would follow a propositional form: because we know that X happens, we should expect Y to occur under Z conditions. 
\end{itemize}


\subsection{What do we do with theory?}

Karl Popper is credited with a \textit{falsificationist} demarcation of science. That is, science is only that which subjects itself to be proven false; scientific knowledge accumulates through the (in)ability to disprove itself. Popper relies on two valid forms of propositions: \textit{modus ponens} and \textit{modus tollens}. \textit{Modus ponens} refers to the constructive argument that, given the statement ``A causes B," we could observe A and B where A is the theory and B is the empirical implication thereof. However, theories are not themselves empirical; that is, you cannot observe a theory. Therefore, we cannot independently make the case for a theory in a positive construction. Conversely, \textit{modus tollens} refers to the destructive argument that, give the same statement as above, a scientist could observe ``not B" and thereby conclude that, given the absence of the empirical implication of a theory, there is also ``not A." 

So what do we do with this? First, we recognize that scientists take a complicated world and try to explain it in a simplified way. That is, we aim to create parsimonious models of the world. Our models are not complete and they do not explain everything. There is no ``theory of everything" for the social world. In this manner, theories are \textit{probabilistic} not deterministic. Second, we should think about where the fault lines of a theory occur. That it, are the falsifying cases central to the theory's core argument or are they more peripheral? Third, we should weigh the evidence that has accumulated for and against the theory. Importantly, there are theories of politics that can become more or less relevant due to coincidental phenomena over time. Finally, we should always evaluate a theory by considering the domain of politics it attempts to explain or not explain, how well the theory explains its domain, and what the theory ``misses" inside or outside of its domain. 



\end{document}
